\chapter{Concluding remarks and closing}%
\label{ch:conclusion}

The present thesis described approaches to the elucidation of reaction
mechanisms in general and the elucidation of specific reaction mechanisms in
particular.
Two computational-experimental collaborations have been presented and a third
purely computational publication detailed the development of an open-source
library and command-line application for automating the investigation of
reaction mechanisms.

ANSWER THE ORIGINAL RESEARCH QUESTIONS.\@

IMPLICATIONS OF THE FINDINGS.\@

LIMITATIONS OF THE RESEARCH.\@

% EXTRA FROM THE CONCLUSIONS
REMOVED BUT SEPARATED AS A ``FUTURES'' LIST BELOW\@.

One could inspect steady-state conditions, in which the concentrations and rates have converged to some final value, search for the most predominant species under catalytic conditions, computationally estimate the overall turn-over frequency, an essential measure of catalytic activity, and understand selectivity of a catalysts towards particular products.
Not only it provides insight into concentration dependencies and comparison with experiment, such framework would serve as a benchmark for hypothesis testing, allowing the prediction of chemical behaviour in a computationally affordable and precise way.
Kinetic models in such a framework can be progressively constructed, with a single pathway added at a time until a match with the experiment, simplifying the workflow, easing the computational burden and increasing efficiency, as was recommended elsewhere~\cite{Jara_z_2019}.
Such an incremental modeling protocol is arguably maximally efficient, as a
minimal set of reactions lead to more easily rationalisable and
``probably less wrong'' models~\cite{Blackmond_2015,Jara_z_2019}.

% REWRITE %
Detailed chemical kinetic modelling of catalytic processes is an emerging area of research. It provides a significantly improved understanding of the reaction mechanisms underlying the manufacture of valuable chemical products. It provides the basis for process improvement and the optimised design of new processes.
% REWRITE %

kinetic modelling

Relevant in the design of novel, compact and highly integrated chemical processes of the future.

% EVEN EXTRA
%%% MICKI %%%
Furthermore, it is possible to probe the sensitivity of the model to individual model parameters, such as rate constants, equilibrium coefficients, and individual adsorbate binding energies. This sensitivity analysis gives information about which steps are rate-limiting and potential descriptors for finding more active or more selective catalysts.
Additionally, Micki allows the user to probe the sensitivity of the model using a built-in sensitivity analysis module.
%%% MICKI %%%

In the future we want to implement the automatic calculation of Marcus theory rate constants, useful in the modelling or hydride transfers~\cite{Nikbin_2012}, and diffusion-controlled rates.
We also want to add automatic kinetic isotope effect calculations.

%  EVEN EVEN MORE FUTURE

Anharmonicity are also envisioned in the future.
Other stuff for the future (already partially or totally implemented but not tested): diffusion controlled reactions and low-lying energy states (through TD-DFT).

In the future: detect host-guest complexes and account for their individual symmetries.
We currently detect reaction symmetries but not for weakly-bound complexes~\cite{Gilson_2010}.

We want to support in the future not only quantum chemistry, but other tools such as molecular dynamics or even experimentally (in a laboratory) determined data.

We pursue a method with which the chemical machines of the future can be efficiently designed, modelled and understood.

Such a tool would allow truly first-principles microkinetic simulations, by taking into account all reasonably relevant effects for chemical reactions in homogeneous media and room or body temperature.
