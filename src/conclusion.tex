\chapter{Concluding remarks and closing}%
\label{ch:conclusion}

The most important contributions presented in this thesis can be summarized as follows:

1. Methods such as density functional theory (DFT), electronic structure calculations of the transition state (TS) or intermediates have been applied to predict chemical reactions from first principles~\cite{Coelho_2019,Oliveira_2020}.

2. The \overreact~software package has been developed for predicting chemical kinetics and simulating microkinetics automatically from first principles~\cite{Schneider2022}.

MY TEXT BELOW.

The present thesis described approaches to the elucidation of reaction
mechanisms in general and the elucidation of specific reaction mechanisms in
particular.
Two computational-experimental collaborations have been presented and a third
purely computational publication detailed the development of an open-source
library and command-line application for automating the investigation of
reaction mechanisms.

Answering in original research questions posited, we provided contributions
to the elucidation of complex reaction mechanisms employing a variety of methodologies,
including the production of a software package for semiautonomous generation of kinetic profiles.
The implications of the contributions are twofold:
(i) facilitate the elucidation of reaction mechanisms in the future, and
(ii) provide ways of making such elucidations faster and more precise.

The limitations of the contributions are related to
(i) effects not taken into account (e.g., solvation entropies),
(ii) bias towards intuitive and well-known reaction motifs, and
(iii) imprecisions due to methodological errors, such as the ones coming from DFT or classical transition state theory.
Those will have to be addressed in future publications.

% FUTURE WORK
Future work is proposed to further develop the \overreact~software package to include the automatic calculation of Marcus theory rate constants,
the use of diffusion-controlled rates,
and automatic kinetic isotope effect calculations.
Additionally, we envision enhance the symmetry module to detect host-guest complexes
and account for their individual symmetries.

MY TEXT BELOW.

In the future, we want to implement (i) the automatic calculation of Marcus theory rate constants,
useful in the modelling or hydride transfers~\cite{Nikbin_2012},
and (ii) diffusion-controlled rates.
We also want to add automatic kinetic isotope effect calculations.

Other future work is envisioned in the symmetry module.
For instance, detect host-guest complexes and account for their individual symmetries.
We currently detect reaction symmetries but not for weakly-bound complexes~\cite{Gilson_2010}.

