\chapter{Concluding remarks and closing}%
\label{ch:conclusion}

% CONTRIBUTIONS

The most important contributions presented in the present thesis,
from the scientific and technological points of view, can be summarized as follows:

\begin{itemize}
	\item \textbf{Electronic structure calculations of the transition state (TS)
		      and/or intermediates have been applied
		      in the elucidation of chemical reactions~\cite{Coelho_2019,Oliveira_2020}}.

	      Approaches to reaction mechanism elucidation were described in general.
	      Computational-experimental collaborative elucidation of two specific reaction mechanisms was described more explicitly,
	      using density functional theory (DFT) calculations~\cite{Coelho_2019,Oliveira_2020}.

	\item \textbf{The~\overreact{}~software package has been developed for predicting chemical reaction kinetics
		      and simulating microkinetics automatically from first principles~\cite{Schneider_2022}}.

	      A third purely computational publication has been included,
	      reporting the development of an open-source library
	      and command-line application for automating the investigation
	      of reaction mechanisms.
	      The package was then applied to a series of different reactions
	      in gas and solvated phases~\cite{Schneider_2022}.
\end{itemize}

% ANSWERS TO THE ORIGINAL RESEARCH QUESTIONS

We hope that the original scientific articles published from the work done
during the present thesis provided
interesting and noteworthy insights in the elucidation of chemical reaction mechanisms,
bridging the gap between classical mechanism elucidation
and modern electronic structure calculations using a variety of methodologies,
including the production of a software package for semiautonomous generation
of kinetic profiles~\cite{Schneider_2022}.
The implications of such contributions are twofold:
%
\begin{itemize*}
	\item the potential to facilitate the elucidation of chemical reaction mechanisms in the future, and
	\item ways of strongly automating such elucidations.
\end{itemize*}

% LIMITATIONS

In terms of limitations, observations should be made regarding
\begin{itemize*}
	\item effects that have not been taken into account in none of the reactions here,
	      such as free-volume solvation entropies, conformational entropies, and others,
	\item screening and explorative bias towards intuitive and well-known reaction motifs,
	      as all pathways were conceived by the authors' chemical intuition, and
	\item imprecisions due to methodological errors, such as potential deficiencies in density functional theory,
	      whose implications can were lightly touched on in the~\overreact{}~article~\cite{Schneider_2022},
	\item limitations of classical transition state theory \emph{per se},
	      which could in principle be overcome by other theories such as variational transition state theory,
	      although with an increased computational cost,
	\item limitations of the applied classical transition state theory corrections,
	      such as quantum tunnelling approximations
	      and the \emph{quasi}-rigid rotor-harmonic oscillator approximation.
\end{itemize*}

% FUTURE WORK

The contributions posed here put in motion a new generation
of models for the elucidation of complex route and emergent reaction mechanisms
based on atomistic simulations and electronic structure calculations,
which in future advancements may introduce the consideration of other effects into the picture
while reducing the limitations listed above.
There is a consensus that such methodological limitations need
to be considered if the outcome of calculations is to be further explored
and their limitations and consequences are to be correctly investigated.
Future work is proposed to include and tackle the following issues:
%
\begin{itemize}
	\item the inclusion of free-volume solvation entropies, for instance
	      by implementing the approach in~\citeauthor{Garza_2019}~\cite{Garza_2019},
	      and conformational entropies, by using results provided by a fast method such as \texttt{xTB}~\cite{Bannwarth_2020},
	\item automatic screening and exploration of chemical reaction pathways
	      by means of machine learning algorithms that have no or minimal inherent human bias,
	\item development of methods for estimating the systematic methodological errors in energies
	      by direct comparison with experimental data, so that a correction be made for similar reactions of interest,
	\item incorporate the numerical calculation of rate constants using different theories,
	      in particular, ones that broaden the applicability of the software,
	      such as Marcus' theory~\cite{Miller_1984,Nobel_1992,Nikbin_2012},
	      necessary in the modelling of reactions
	      such as electron~\cite{Miller_1984} and hydride~\cite{Nikbin_2012} transfers,
	\item adjustment of reaction rate constants for diffusion-controlled regimes,
	\item use of improved quantum tunnelling corrections based on the shape of the calculated potential energy surface.
\end{itemize}

Finally, two further features are envisioned for addition in~\overreact{},~but don't necessarily mitigate the issues described above.
The first is the automatic calculation of kinetic isotope effects in general.
The other is the extension of the symmetry module to more complicated host-guest complexes
in order to account for their individual symmetries:
we currently detect point-group and reaction symmetries but not internal symmetries
of weakly bound complexes~\cite{Gilson_2010}.
