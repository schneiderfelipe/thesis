\chapter{Concluding remarks and closing}%
\label{ch:conclusion}

The present thesis described approaches to the elucidation of reaction
mechanisms in general and the elucidation of specific reaction mechanisms in
particular.
Two computational-experimental collaborations have been presented and a third
purely computational publication detailed the development of an open-source
library and command-line application for automating the investigation of
reaction mechanisms.

Answering in original research questions posited, we provided contributions
to the elucidation of complex reaction mechanisms employing a variety of methodologies,
including the production of a software package for semiautonomous generation of kinetic profiles.
The implications of the contributions are twofold:
(i) facilitate the elucidation of reaction mechanisms in the future, and
(ii) provide ways of making such elucidations faster and more precise.

The limitations of the contributions are related to
(i) effects not taken into account (e.g., solvation entropies),
(ii) bias towards intuitive and well-known reaction motifs, and
(iii) imprecisions due to methodological errors, such as the ones coming from DFT or classical transition state theory.
Those will have to be addressed in future publications.

% TODO:
% EXTRA FROM THE CONCLUSIONS
% REMOVED BUT SEPARATED AS A ``FUTURES'' LIST BELOW.\@

One could inspect steady-state conditions, in which the concentrations and rates have converged to some final value, search for the most predominant species under catalytic conditions, computationally estimate the overall turn-over frequency, an essential measure of catalytic activity, and understand selectivity of a catalysts towards particular products.
Not only it provides insight into concentration dependencies and comparison with experiment, such framework would serve as a benchmark for hypothesis testing, allowing the prediction of chemical behaviour in a computationally affordable and precise way.
Kinetic models in such a framework can be progressively constructed, with a single pathway added at a time until a match with the experiment, simplifying the workflow, easing the computational burden and increasing efficiency, as was recommended elsewhere~\cite{Jara_z_2019}.
Such an incremental modeling protocol is arguably maximally efficient, as a
minimal set of reactions lead to more easily rationalisable and
``probably less wrong'' models~\cite{Blackmond_2015,Jara_z_2019}.
