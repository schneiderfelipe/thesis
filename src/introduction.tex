\chapter{Introduction}%
\label{ch:introduction}

Chemical reactions are more important than ever.
And today's economy wants chemical reactions to go fast:
catalysts are employed in the production of over 80\% of all chemical
goods of industrial
importance~\cite{Breakthrough_Catalyst,GlobalCatalystMarket}.
In 2014, the global market share of catalysts was evaluated in US\$ 33.5
billion~\cite{GlobalCatalystMarket}, and growing~\cite{GlobalCatalystMarket},
with an annual share in the global economy estimated to be worth US\$ 10
trillion~\cite{GlobalCatalystMarket}.

With the advent of environmental and energetic challenges~\cite{Bertozzi_2016},
there is a global need for precise prediction of chemical reactions from first
principles.
Various techniques have been applied for this purpose~\cite{Cao2022}.
And there is room for improvement: currently industrially-employed catalysts do
not reach near the throughput, atomic economy, selectivity and efficiency that
are so frequently attained by natural
enzymes~\cite{Catalysis_in_Chemistry_and_Enzymology}.

If we have to compete with natural enzymes' billion years worth of evolution,
we need to be able to predict chemical reactions from first principles with a
high degree of accuracy.
As such, there is demand for research to go beyond simple Gibbs' free energy
diagrams and consider important aspects of complex chemical reaction networks
such as concentrations and quantum tunneling.

The present part presents an overview of methods for predicting chemical
kinetics and thermodynamics from first principles.
It starts with a brief introduction to the general state of the art in quantum
chemistry and its applications to the study of chemical reactions.
It then goes on to describe the methods employed in \overreact, a software
package for predicting chemical kinetics and simulating microkinetics
automatically from first principles.
