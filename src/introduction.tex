\chapter{Introduction}%
\label{ch:introduction}

% TODO:
% Detailed chemical kinetic modelling of catalytic processes is an emerging area of research.
% It provides a significantly improved understanding of the reaction mechanisms underlying the manufacture of valuable chemical products.
% It provides the basis for process improvement and the optimised design of new processes.
% Relevant in the design of novel, compact and highly integrated chemical processes of the future.

Given the advent of environmental and energetic challenges~\cite{Bertozzi_2016},
the rational design of chemical reactions is more important than ever.
And today's economy demands fast and efficient chemical reactions:
catalysts are employed in the production of over 80\% of all chemical
goods of industrial
importance~\cite{Breakthrough_Catalyst,GlobalCatalystMarket}.
In fact, the 2014 global market share of catalysts was evaluated in US\$~33.5 billion,
with an annual share in the global economy estimated to be worth US\$~10
trillion~\cite{GlobalCatalystMarket}.
As such, there has been a global urgency for precise prediction of chemical reactions from first principles~\cite{Kitchin_2012}.

Various techniques have been applied for this purpose~\cite{Houk_2014,Chin_2022,Cao2022}.
Computer-aided strategies, such as density functional theory (DFT),
provide theoretical models for mapping out even complex reaction pathways from initial starting material~\cite{Maeda_2011,Simm_2017,Rappoport_2019}.
Additionally, electronic structure calculations of the transition state (TS) or intermediates can be used to determine the reaction rate constants,
which are key in the interpretation of experimental data~\cite{Plata_2015,Santoro_2016,Coelho_2019,Oliveira_2020}.
Recent studies have also explored the use of machine learning algorithms
to predict chemical reaction rate constants, providing a useful tool for
accelerating reaction catalyst design (CITATION).
Finally, quantum mechanics and statistical mechanics techniques,
such as molecular dynamics and Monte Carlo methods,
can be employed to obtain thermodynamic and kinetic information for many reactions~\cite{Wang_2014}.

But there is natural room for improvement:
despite the recent great efforts~\cite{Peng_2016}, currently industrially-employed catalysts do
not reach near the throughput, atomic economy, selectivity and efficiency that
are so frequently attained by natural
enzymes~\cite{Catalysis_in_Chemistry_and_Enzymology}.
If are to rationally design chemical reactions on par with natural enzymes' billion years worth of evolution,
we need to be able to predict chemical reactions from first principles with a
high degree of accuracy.
Such predictions must go beyond simple Gibbs' free energy
diagrams and consider important aspects of complex chemical reaction networks
such as concentrations, quantum tunneling, diffusion effects and others~\cite{Besora_2018}.

The present part presents an overview of methods for predicting chemical
kinetics and thermodynamics from first principles.
It starts with a brief introduction to the general state of the art in quantum
chemistry and its applications to the study of chemical reactions.
It then goes on to describe the methods employed in
\overreact~\cite{Schneider2022}, a software
package for predicting chemical kinetics and simulating microkinetics
automatically from first principles.

We pursue a method with which the chemical machines of the future can be efficiently designed, modelled and understood.
Such a tool would allow truly first-principles microkinetic simulations,
by taking into account all reasonably relevant effects for chemical reactions in homogeneous media and room or body temperature.
