\chapter{Introduction}%
\label{ch:introduction}

Given the advent of environmental and energetic challenges~\cite{Bertozzi_2016},
chemical reaction rational design is more important than ever.
And today's economy demands fast chemical reactions:
catalysts are employed in the production of over 80\% of all chemical
goods of industrial
importance~\cite{Breakthrough_Catalyst,GlobalCatalystMarket}.
In fact, the 2014 global market share of catalysts was evaluated in US\$~33.5 billion,
with an annual share in the global economy estimated to be worth US\$~10
trillion~\cite{GlobalCatalystMarket}.

As such, there has been a global urgency for precise prediction of chemical reactions from first principles.
Various techniques have been applied for this purpose~\cite{Cao2022}.
% TODO: WHICH TECHNIQUES?\@ FOCUS ON BAREBONES TECHNIQUES, MICROKINETICS AND STUFF COMES LATER.\@

But there is natural room for improvement:
currently industrially-employed catalysts do
not reach near the throughput, atomic economy, selectivity and efficiency that
are so frequently attained by natural
enzymes~\cite{Catalysis_in_Chemistry_and_Enzymology}.
If are to rationally design chemical reactions on par with natural enzymes' billion years worth of evolution,
we need to be able to predict chemical reactions from first principles with a
high degree of accuracy.
Such predictions must go beyond simple Gibbs' free energy
diagrams and consider important aspects of complex chemical reaction networks
such as concentrations, quantum tunneling, diffusion effects and others.
% TODO: INSERT CITATIONS AND TALK ABOUT EFFORTS TOWARDS THIS GOAL.\@

% TODO:
% THIS INTRODUCTORY SECTION SEEMS SUPERFICIAL AND DOES NOT PROVIDE AN IDEA OF THE
% STATE OF THE ART THAT THE SUBJECT DEMANDS.\@
% IT NEEDS TO BE EXPANDED, HIGHLIGHTING THE IMPACTS, ACHIEVEMENTS AND GAPS YET TO
% BE FILLED.\@
% IT NECESSARILY CONVERSES WITH THE ABOVE, AS WE NEED TO FILL THE GAP WITH THIS THESIS,
% SO FILL THE ABOVE FIRST.\@
The present part presents an overview of methods for predicting chemical
kinetics and thermodynamics from first principles.
It starts with a brief introduction to the general state of the art in quantum
chemistry and its applications to the study of chemical reactions.
It then goes on to describe the methods employed in
\overreact~\cite{Schneider2022}, a software
package for predicting chemical kinetics and simulating microkinetics
automatically from first principles.
