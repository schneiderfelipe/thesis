\chapter*{Introduction}%
\label{ch:introduction}

The development of catalysis in general, and fine-tuned control of chemical reactions in particular,
has truly shaped the world we live in today.
A striking example is the development of the Haber-Bosch process,
a ``synthesis of ammonia from its elements''~\cite{Nobel_1918},
which Otto~Haber had developed in the early 1900s~\cite{Erisman_2008,Lewis_2015},
receiving the Nobel Prize for it in 1918~\cite{Nobel_1918}.
Carl Bosch later improved the industrial reaction through high pressure methods,
which awarded him the Nobel Prize in 1931~\cite{Nobel_1931}.
Together, development and precise control were so impressive that it is often used
as an example of the geological impact of human actions, a key defining milestone for
the era we live in, the Anthropocene~\cite{Smil_1999,Canfield_2010,Lewis_2015}.
%
\begin{citacao}
	Human activity profoundly affects the environment,
	from Earth’s major biogeochemical cycles to the evolution of life.
	For example, the early-twentieth-century invention of the Haber-Bosch process,
	which allows the conversion of atmospheric nitrogen to ammonia for use as fertilizer,
	has altered the global nitrogen cycle so fundamentally
	that the nearest suggested geological comparison refers to events
	about~2.5~billion~years~ago~\cite{Lewis_2015}.
\end{citacao}

More than a century later,
with the advent of climate change and other environmental and energetic challenges~\cite{Bertozzi_2016},
the rational design of chemical reactions is more important than ever before.
This is not only due to the impact of human actions,
but also \emph{precisely} because of it:
humans have the decision power to start making positive impact on the environment we live in
by transitioning to a greener economy.
On the other hand, today's economy demands fast and efficient chemical reactions:
catalysts are employed in the production of over 80\% of all chemical
goods of industrial
importance~\cite{Breakthrough_Catalyst,GlobalCatalystMarket},
and make up a considerable piece of human activity.
In fact, the 2014 global market share of catalysts was evaluated in US\$~33.5 billion,
with an annual share in the global economy estimated to be worth US\$~10
trillion~\cite{GlobalCatalystMarket}.

The apparent conflict between these two requirements can be reconciled by
the rational design of green,
fast and efficient chemical transformations and catalysts,
in particular, with precise prediction of chemical reactions from first principles~\cite{Kitchin_2012}.
The ability to accurately predict and account for the kinetics and thermodynamics associated
with chemical procedures remains an immense challenge,
yet a central objective in various scientific sectors,
with broad implications for the future.
Since complex reaction networks can critically inform our interpretation of experimental results,
and enable the design of efficient chemical production processes,
a broad interest in efficient modelling of the corresponding prediction methods exists.
Such rational modelling provides the basis for process improvement and the optimised design of new processes
that is demanded today.

Various techniques have been applied for this purpose~\cite{Houk_2014,Chew_2020,Chin_2022,Cao2022}.
Computer-aided strategies,
such as density functional theory (DFT),
provide theoretical models for mapping out even complex reaction pathways from initial starting material~\cite{Maeda_2011,Simm_2017,Rappoport_2019}.
Additionally,
electronic structure calculations of the transition state (TS) or intermediates can be used to determine the reaction rate constants,
which are key in the interpretation of experimental data~\cite{Plata_2015,Santoro_2016,Coelho_2019,Oliveira_2020}.
Recent studies have also explored the use of machine learning algorithms
to predict chemical reaction rate constants,
providing a useful tool for
accelerating reaction catalyst design~\cite{Komp_2022,Tu_2022}.
Finally,
quantum mechanics and statistical mechanics techniques,
such as molecular dynamics and Monte Carlo methods,
can be employed to obtain thermodynamic and kinetic information for many reactions~\cite{Wang_2014}.

But there is natural room for improvement:
despite the recent great efforts~\cite{Peng_2016},
currently industrially-employed catalysts do
not reach near the throughput,
atomic economy,
selectivity and efficiency that
are so frequently attained by natural
enzymes~\cite{Catalysis_in_Chemistry_and_Enzymology}.
If we are to rationally design chemical reactions on par with natural enzymes' billion years worth of evolution,
we need to be able to predict chemical reactions from first principles with a
high degree of accuracy.
Such predictions must go beyond simple Gibbs' free energy
diagrams and consider important aspects of complex chemical reaction networks
such as concentrations,
quantum tunnelling,
diffusion effects and others~\cite{Besora_2018}.
Such modelling provides a significantly improved understanding of the reaction mechanisms underlying the manufacture of valuable chemical products.

Although much of the chemical and industrial breakthroughs were serendipitous,
and they were largely based on inspiration, intuition and empirical testing,
definition of rational principles of design have involved an increasing amount of computational
chemistry~\cite{Kitchin_2012}.
Indeed, nowadays, the role of computational chemistry for industrial applications can
hardly be replaced by any other means,
as it is often used to boost experimental efforts.
Understanding what the global energy landscape of the reaction network looks like,
and how it depends on the reaction conditions is thus instrumental in providing a guiding hand towards making efficient,
industrially-viable improvements.
Detailed chemical kinetic modelling of catalytic processes is an emerging area of research and key stepping stone for the future.

\section*{Scope}%
\label{sec:scope}

This thesis attempts to formalize a review of methods for predicting chemical
kinetics and thermodynamics from first principles.
It starts with a brief introduction to the general state of the art in quantum
chemistry and state-of-the-art applications to the study of chemical reactions.
It then goes on to describe the methods employed in
\overreact~\cite{Schneider2022},
a software
package for predicting chemical kinetics and simulating microkinetics
automatically from first principles.

We pursue methods for predicting and designing reaction mechanisms
and equilibrium schemes for the chemical machines of the future,
with the aim to provide an efficient way to design, understand, evaluate and optimize reactions and their conditions.
The focus is in methods and techniques that both aid experimental work and can be validated by it.
Such framework allows first-principles microkinetic simulations,
by predicting the reaction mechanism from first principles,
taking into account most relevant effects influencing
chemical reactions in homogeneous media at both room and body temperatures,
impacting the design of novel,
compact and highly integrated future chemical processes.

\section*{Outline}%
\label{sec:outline}

The present thesis is organized in two parts.
\cref{part:art-theory}
encompasses a brief overview of the current state of the art in the area of
computational elucidation of reaction mechanisms.
\cref{part:publications} is divided into three chapters, each of which
presents an application or development to certain problems in the field.

The chapters in~\cref{part:art-theory} provide a detailed account of the
field.
\cref{ch:lit-review} summarizes the broad theme of reaction mechanism
elucidation by computational means and its promising applications.
\cref{ch:methods} describes the methods available to computationally
investigate reaction mechanisms and, in particular, the methods used in the
present thesis.
In~\cref{sec:overreact-methods}, attention is paid to the design
of \overreact~\cite{Schneider2022,overreact2021zenodo}, a software package developed by the
author for the purpose of automating the investigation of reaction mechanisms
in general.

The chapters in~\cref{part:publications} revolve around papers that have
been published in the field and co-authored by the author.
The applications are intertwined in the sense
that they contribute to the elucidation of more complex reaction mechanisms.
\cref{ch:paper1} deals with
computational-experimental collaboration on the elucidation
of the Palladium(II)-mediated uncaging reaction of propargylic substrates,
with applications to the activation of prodrug molecules.\@
It was published in
\emph{\fullcite{Coelho_2019}}.
\cref{ch:paper2} provides an account of
another computational-experimental collaboration on prodrug activation,
this time on the Platinum(II)-triggered bond-cleavage
of pentynoyl amide and \ce{N}-propargyl handles.\@
The relevant publication is
\emph{\fullcite{Oliveira_2020}}.
Finally,~\cref{ch:paper3} introduces the
\overreact{} package, a software that performs microkinetic simulations
from first principles in an automated way.\@
It was published under
\emph{\fullcite{Schneider2022}}.
In all three chapters, the papers are presented in full text.
These encompass what the author believes to be his most relevant
contributions to the field, as detailed in~\cref{sec:major-contributions}.
Other minor contributions that tangentially relate to the field are commented
on in the text (see~\cref{sec:minor-contributions}).
Furthermore, works not directly related to the
field, but were nevertheless published during the course of the thesis are
presented in~\cref{ch:all-works}.

A brief, closing perspective is given in~\cref{ch:conclusion}, where
the author concludes the thesis with an eye on what he believes to be the
future of the field.
Finally, the present thesis contains two appendices.
Together with the already mentioned~\cref{ch:all-works}, which lists all
the contributions co-authored,~\cref{ch:tutorial} provides a practical but
short walkthrough over technical issues and methodologies developed during the
thesis in the context of the main field, but that are rarely mentioned in the
literature.
Since most of the computational work in obtaining transition state geometries
is still performed by trial and error, the author believes that~\cref{ch:tutorial} is an important contribution to the field as well.
