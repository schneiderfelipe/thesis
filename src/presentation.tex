\chapter{Presentation and structure}%
\label{ch:presentation}

THESIS STRUCTURE.\@
CORE CHAPTERS.\@
WHAT IS IN EACH CHAPTER?\@

The present thesis is organized in two parts. \autoref{part:art-theory}
encompasses a brief overview of the current state of the art in the area of
computational elucidation of reaction mechanisms.
\autoref{part:publications} is divided into three chapters, each of which
presents an application or development to certain problems in the field.

The chapters in \autoref{part:art-theory} provide a detailed account of the
field.
\autoref{ch:introduction} summarizes the broad theme of reaction mechanism
elucidation by computational means and its promising applications.
\autoref{ch:methods} describes the methods available to computationally
investigate reaction mechanisms and, in particular, the methods used in the
present thesis.
In \autoref{sec:overreact-methods}, attention is paid to the design
of \overreact~\cite{Schneider2022,overreact2021zenodo}, a software package developed by the
author for the purpose of automating the investigation of reaction mechanisms
in general.

The chapters in \autoref{part:publications} revolve around papers that have
been published in the field and co-authored by the author.
\autoref{ch:paper1} deals with
EXPLAIN PAPER 1.\@
It was published in \citetext{Coelho2019}.
\autoref{ch:paper2} provides an account of
EXPLAIN PAPER 2.\@
The relevant publication is \citetext{Oliveira2020}.
Finally, \autoref{ch:paper3} introduces the
EXPLAIN PAPER 3.\@
It was published under \citetext{Schneider2022}.

In all three chapters, the papers are presented in full text.
These encompass what the author believes to be his most relevant
contributions to the field, as detailed in \autoref{sec:major-contributions}.
Other minor contributions that tangentially relate to the field are commented
on in the text (see \autoref{sec:minor-contributions}).
Furthermore, works not directly related to the
field but were nevertheless published during the course of the thesis are
presented in
\autoref{ch:all-works}.

A brief, closing perspective is given in \autoref{ch:conclusion}, where
the author concludes the thesis with an eye on what he believes to be the
future of the field.

The present thesis contains two appendices.
Together with the already mentioned \autoref{ch:all-works}, which lists all
the contributions co-authored, \autoref{ch:tutorial} provides a practical but
short walkthrough over technical issues and methodologies developed during the
thesis in the context of the main field, but that are rarely mentioned in the
literature.
Since most of the computational work in obtaining transition state geometries
is still performed by trial and error, the author believes that
\autoref{ch:tutorial} is an important contribution to the field as well.
