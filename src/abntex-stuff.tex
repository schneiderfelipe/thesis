% ----------------------------------------------------------
% PARTE
% ----------------------------------------------------------
\part{Preparação da pesquisa}
% ----------------------------------------------------------

\subimport{.}{commands}
% ---

\chapter{Conteúdos específicos do modelo de trabalho acadêmico}%
\label{ch:trabalho_academico}

\section{Quadros}

Este modelo vem com o ambiente \texttt{quadro} e impressão de Lista de quadros
configurados por padrão. Verifique um exemplo de utilização:

\begin{quadro}[htb]
\caption{%
\label{qua:exemplo}Exemplo de quadro}
\begin{tabular}{cccc}
    \toprule
    \textbf{Pessoa} & \textbf{Idade} & \textbf{Peso} & \textbf{Altura} \\
    \midrule
    Marcos          & 26             & 68            & 178             \\
    Ivone           & 22             & 57            & 162             \\
    \ldots          & \ldots         & \ldots        & \ldots          \\
    Sueli           & 40             & 65            & 153             \\
    \bottomrule
\end{tabular}
\fonte{Autor.}
\end{quadro}

Este parágrafo incrível apresenta como referenciar o quadro no texto, requisito
obrigatório da ABNT.\@
Primeira opção, utilizando \texttt{autoref}: Ver o~\autoref{qua:exemplo}.
Segunda opção, utilizando \texttt{ref}: Ver o Quadro~\ref{qua:exemplo}.

% ----------------------------------------------------------
% PARTE
% ----------------------------------------------------------
\part{Referenciais teóricos}
% ----------------------------------------------------------
