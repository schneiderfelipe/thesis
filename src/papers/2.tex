\chapter{Paper II:\@
  Pt-triggered bond-cleavage of pentynoyl amide
  and n-propargyl handles for drug-activation
 }%
\label{ch:paper2}

\citetext{Oliveira2020}

Here we present a new \ce{Pt}-based smart drug was developed for cancer therapy~\cite{Oliveira2020}.
We studied a biorthogonal \ce{Pt}-complex for prodrug applications whose controlled activation consists
of a water-directed bond-cleavage reaction, releasing a therapeutical protecting group.
This application to metallocatalysis was done in a joint computational-experimental collaboration.

In order for such prodrugs to work, they must be selective to the type of tumor
being treated, as well as having low toxicity.
Metal-mediated decaging prodrugs offer the possibility of being catalytic,
allowing them to be used in substoichiometric amounts, which thus reduces the
required dosage.

Other metals were already studied in the literature, including
\ce{Pd}~\cite{Coelho2019}, \ce{Ru} and \ce{Au}.
\ce{Pt} was chosen since other complexes such as Cisplatin are widely used in
the clinic, having short half-life in humans, not being present in human
biology, and having the ability to concentrate in the tumor site.

We introduce a new biorthogonal cleavage reaction catalyzed by platinum for
prodrug activation.
We demonstrate that pentynoyl tertiary amide and \ce{N}-propargyl amide handles
in small-molecule drugs decage succesfully in both aqueous and cell media under
nontoxic quantities of \ce{Pt} salts.
Studies in zebrafish models for the treatment of colorectal cancer showed
promissing results.

SCHEME I from the paper.
Also, FIGURE I (for the reaction).

















HOW IS THE RESEARCH DESIGNED?\@

WHY IT IS DESIGNED THIS WAY?\@




WHAT DOES THE LITERATURE SAY ABOUT THIS?\@

IS THE LITERATURE WELL STABLISHED?\@
IS IT DIVIDED?\@

HOW DOES THE RESEARCH FIT THE BIGGER PICTURE?\@

HOW DOES THE RESEARCH CONTRIBUTE SOMETHING ORIGINAL?\@

HOW DOES THE METHODOLOGY OF PREVIOUS STUDIES HELP YOU DEVELOP YOUR OWN?\@








WHY IS THIS WORTH INVESTIGATING?\@
HOW IMPORTANT IS THIS?\@
HOW IS THIS ORIGINAL?\@

WHAT WERE MY RESEARCH AIMS?\@

WHAT IS THE SCOPE OF MY STUDY?\@
WHAT I COVERED AND DIDN'T COVER?\@

WHICH METHODS WERE USED?\@





\section{Background and motivation}

PRESENTATION OF THE WORK.\@

Given that this work was featured in the journal's cover, it has received some local media coverage~\cite{noticias-da-ufsc2020}.

DESCRIPTION OF THE WORK.\@

OBJECTIVES OF THE WORK.\@

INTERPRETATION AND MEANING OF THE WORK.\@

MAIN FINDINGS.\@

RESULTS IN RELATION TO THE RESEARCH QUESTIONS.\@

\section{Paper}

The publication can be read in full next.

\includepdf[pages=-]{pubs/oliveira2020-paper2.pdf}
