\chapter{Paper II:\@
  Pt-triggered bond-cleavage of pentynoyl amide
  and n-propargyl handles for drug-activation
 }%
\label{ch:paper2}

\citetext{Oliveira2020}

Here we present a new \ce{Pt}-based smart drug was developed for cancer therapy~\cite{Oliveira2020}.
We studied a biorthogonal \ce{Pt}-complex for prodrug applications whose controlled activation consists
of a water-directed bond-cleavage reaction, releasing a therapeutical protecting group.
This application to metallocatalysis was done in a joint computational-experimental collaboration.

In order for such prodrugs to work, they must be selective to the type of tumor
being treated, as well as having low toxicity.
Metal-mediated decaging prodrugs offer the possibility of being catalytic,
allowing them to be used in substoichiometric amounts, which thus reduces the
required dosage.

Other metals were already studied in the literature, including
\ce{Pd}~\cite{Coelho2019}, \ce{Ru} and \ce{Au}.
\ce{Pt} was chosen since other complexes such as Cisplatin are widely used in
the clinic, having short half-life in humans, not being present in human
biology, and having the ability to concentrate in the tumor site.

We introduce a new biorthogonal cleavage reaction catalyzed by platinum for
prodrug activation.
We demonstrate that pentynoyl tertiary amide and \ce{N}-propargyl amide handles
in small-molecule drugs decage succesfully in both aqueous and cell media under
nontoxic quantities of \ce{Pt} salts.
Studies in zebrafish models for the treatment of colorectal cancer showed
promising results.

SCHEME I from the paper.
Also, FIGURE I (for the reaction).
FIGURE II(f) has some info for the calculated mechanism.

As a highlight, it was experimentally found that water was required to generate
the active catalyst, which means activation of metal-chloride bonds by water,
which is rare~\cite{Vidal_2018}.

THERE ARE KINETIC CONSTANTS IN THE PAPER.\@

Calculations indicate that the reaction takes place through a stepwise process
starting with coordination of the substrate molecule to \ce{Pt(II)}, followed by
an intramolecular attack of the carbonyl oxygen to the pentynoyl moiety,
producing a five-membered ring intermediate.
From this intermediate, different decomposition pathways were explored.
The one with lowest energy encompasses a hydration of the intermediate, which
then further decomposes to liberate the free amine.
A subsequent hydrolysis recovers the metal complex.

SI MOVIE AND FIGURES 20 and 21

The intermediate \ce{CS_0} identified by LC-MS was determined by the computational
methodology.
The difference in activation free energy between both substrates was found to
be around 2.8~kcal/mol.

A new decaging reaction of alkynes with platinum complexes were developed,
aiming at the release secondary amines from stable tertiary amides.
The reaction happens successfully both in cell cultures and living organisms.
We showed that the reaction takes place by a platinum-mediated intramolecular
cyclization mechanism and that water works as the metal-activating agent.
Our computational model matches well the thorough LC-MS characterization of the
reaction intermediates.
The reaction was further extended to work with \ce{N}-propargyl groups with
comparable efficacies to palladium-mediated depropargylation~\cite{Coelho2019}.

In the group of Prof.~Bernardes, it was adapted for the synthesis of a
noninternalizing ADC (WHAT IS THIS?), resulting in drug release upon
administration of platinum complexes in cancer cells.
It was also demonstrated to work effectively in colorectal cancer zebrafish
xenograft model employing nontoxic amounts of CisPt in order to activate a
prodrug of the anticancer agent 5-FU (WHAT IS THIS AGAIN?), which effectively
led to significant tumor reduction \emph{in vivo}.

This adds a significant decaging strategy for the toolbox of chemical biology
applications.
It reaction is effective in aqueous systems encompassing high concentrations of
salts with high yields and reaction rates, similar to the standard palladium
decaging metal (WHAT?).
A downside is the sucetibility of the reaction towards the presence of
nucleophiles, which leads to one order of magnitude slower rates.
It is also compatible with celullar environments, although the presence of
biomolecules/nucleophiles reduces overall yields considerably.
These results pave the way for future developments of platinum-mediated
decaging reactions.

HOW IS THE RESEARCH DESIGNED?\@

WHY IT IS DESIGNED THIS WAY?\@

WHAT DOES THE LITERATURE SAY ABOUT THIS?\@

IS THE LITERATURE WELL STABLISHED?\@
IS IT DIVIDED?\@

HOW DOES THE RESEARCH FIT THE BIGGER PICTURE?\@

HOW DOES THE RESEARCH CONTRIBUTE SOMETHING ORIGINAL?\@

HOW DOES THE METHODOLOGY OF PREVIOUS STUDIES HELP YOU DEVELOP YOUR OWN?\@

WHY IS THIS WORTH INVESTIGATING?\@
HOW IMPORTANT IS THIS?\@
HOW IS THIS ORIGINAL?\@

WHAT WERE MY RESEARCH AIMS?\@

WHAT IS THE SCOPE OF MY STUDY?\@
WHAT I COVERED AND DIDN'T COVER?\@

WHICH METHODS WERE USED?\@

\section{Background and motivation}

PRESENTATION OF THE WORK.\@

Given that this work was featured in the journal's cover, it has received some local media coverage~\cite{noticias-da-ufsc2020}.

DESCRIPTION OF THE WORK.\@

OBJECTIVES OF THE WORK.\@

INTERPRETATION AND MEANING OF THE WORK.\@

MAIN FINDINGS.\@

RESULTS IN RELATION TO THE RESEARCH QUESTIONS.\@

\section{Paper}

The publication can be read in full next.

\includepdf[pages=-]{pubs/oliveira2020-paper2.pdf}
