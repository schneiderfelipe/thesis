\chapter{Paper II:\@
  Pt-triggered bond-cleavage of pentynoyl amide
  and n-propargyl handles for drug-activation
 }%
\label{ch:paper2}

\fullcite{Oliveira_2020}

% Abstract
Platinum complexes can be used to rapidly trigger controlled release and activation of drugs in cells and tumors.

% Introduction
Drugs are targeted to tumor cells using antibody-drug conjugates, with release triggered by external triggers to avoid variable biology among subjects.
Noninternalizing ADCs may have tumor-selectivity and short in vivo retention issues, affecting efficacy and depending on tumor size, may lack a therapeutic response.
Contrast that to internalizing ADCs which accumulate inside tumor cells.
Metal-mediated decaging of prodrugs can result in reduced toxicity and tumor growth inhibition.
Palladium-mediated decaging is majorly studied for prodrug activation, facilitated by metals like ruthenium and gold, with copper as the most recent addition.
Platinum-based catalysts may potentially be used to activate prodrugs in cancer therapy and is an area that has not yet been sufficiently explored.
CisPt is a chemotherapy drug used to treat 20\% of cancer patients; highly reactive, accumulates in the tumor, not present in biology.
Tumor concentration estimated 0.83–12.3 μM.
We demonstrated successful decaging of platinum-catalyzed cleavage reactions for prodrug activation and drug release from an ADC in cancer cells, plus application to a zebrafish xenograft model.

% Results and discussion
% - Engineering of a Platinum-Mediated Decaging Reaction
Pt and Au complexes share similar reactions, which were adapted for bioorthogonal decaging at room temperature in aqueous solutions.
Cyclization of 4-pentynoic acid was observed, as well as the release of morpholine from a terminal propargyl carbamate.
NMR spectroscopy used to study the decaging of tertiary amide compound 4a with Pt-mediated cyclization and release of morpholine 6a.
Decaging of tertiary amides to release secondary amines is possible and yields good results with substoichiometric amounts of metal catalysts.
Decaging of stable protected tertiary amides was performed with low amount of platinum complexes in water or open air without extreme temperatures/complex ligands, but some conversion was not observed due to side reactions.
% - Mechanistic and Kinetic Studies of the Platinum-Mediated Decaging Reaction
The platinum decaging reaction was monitored by an increase in fluorescence, with an initial steady state followed by an increase when aqua complexes were formed.
Platinum reaction studied with fluorogenic probes to cleave alkyne-containing molecules; rate constants determined in presence of metal poisons with model substrate; calculated mechanism with functional revPBE and DLPNO-CCSD(T).
K2PtCl4, Pd(OAc)2 and CisPt complexes all yielded decaged product with high efficiency in different conditions, including physiological.
The rate constant of a platinum complex-assisted reaction of decaging pentynoyl tertiary amides was determined to be 0.230 ± 0.004 M–1 s–1 and 0.080 ± 0.002 M–1 s–1 for different catalysts.
Adding CS2 and EDTA had no effect on Pt(0) and Pt(II) decaging reaction rates respectively.
Computational studies suggest a stepwise reaction pathway with intramolecular attack of the Pt-coordinated substrate giving a five-membered ring intermediate, leading to hydration, decomposition and release of the amine product.
% - Extending the Decaging Reaction to N-Propargyl Group
Investigated the possibility of using N-propargyl groups introduced on drugs of interest for prodrug activation with platinum triggers; reactions were slower than with pentynoyl amides; reactions were complete in 1-2.5 hrs in E3 media and 14h in DMEM.
% - Platinum-Mediated Decaging in Living Cells
Platinum-mediated depropargylation of pentynoyl amide derivative MMAE and N-propargyl 5-fluorouracil (pFU) derivatives was seen in cell culture, with increases in toxicity.
HeLa cells were tested for toxicity with MMAE-am 11a, pFU 12b and their platinum-mediated decaging derivatives; with significant differences revealed.
The use of low concentrations of K2PtCl4 and CisPt is possible but leads to modest increases in toxicity and lower conversions in the presence of nucleophiles.
% - Platinum Decaging of ADC
A caging group for chemically controlled drug release was adapted for an ADC, using MMAE as a common payload coupled to a carbonyl acrylic bioconjugation handle.
F 16-14 ADC was modified with MMAE and purified then decaged with platinum, resulting in a statistically significant decrease in cell viability.
Testing of linker's platinum decaging showed release of MMAE with complete consumption of 14.
ADC was successfully made with a noninternalizing antibody.
Decaging with platinum complex released MMAE in HeLa cells and model protein.
% - Cisplatin-Mediated Prodrug Decaging in Vivo
Zebrafish larvae xenograft model was used to test the efficacy of pFU and its combinatory effect with CisPt in vivo.
Zebrafish larvae were exposed to Fluorogenic probe 9 and randomly distributed into two conditions before being imaged with a Confocal.
Efficacy of CisPt depropargylation was measured following maximum tolerated concentration assessment and HCT116 xenograft treatment.
HCT116 cells were injected into Tg(Fli1:eGFP) zebrafish larvae, treated and analyzed for proliferation, apoptosis and tumor size.
Statistical analysis shows significant results.
The combined effect of the anticancer prodrug pFU with CisPt was more pronounced than the combination of 5-FU with CisPt, in regards to both proliferation and tumor size.

% Conclusions
A new decaging reaction using platinum complexes releases secondary amines from tertiary amides in mammalian cells and living organisms.
Platinum-mediated cleavable reaction can be accomplished in aqueous systems with high yields, but is susceptible to nucleophiles making it slower and less effective in cellular environments.
Our work revealed the instability of platinum complexes for prodrug activation in physiological/biological conditions, so further studies are needed for potential in vivo applications.

MY TEXT GOES BELOW.

Here we present a new \ce{Pt}-based smart drug was developed for cancer therapy~\cite{Oliveira_2020}.
We studied a biorthogonal \ce{Pt}-complex for prodrug applications whose controlled activation consists
of a water-directed bond-cleavage reaction, releasing a therapeutical protecting group.
This application to metallocatalysis was done in a joint computational-experimental collaboration.

In order for such prodrugs to work, they must be selective to the type of tumor
being treated, as well as having low toxicity.
Metal-mediated decaging prodrugs offer the possibility of being catalytic,
allowing them to be used in substoichiometric amounts, which thus reduces the
required dosage.

Other metals were already studied in the literature, including
\ce{Pd}~\cite{Coelho_2019}, \ce{Ru} and \ce{Au}.
\ce{Pt} was chosen since other complexes such as Cisplatin are widely used in
the clinic, having short half-life in humans, not being present in human
biology, and having the ability to concentrate in the tumor site.

We introduce a new biorthogonal cleavage reaction catalyzed by platinum for
prodrug activation.
We demonstrate that pentynoyl tertiary amide and \ce{N}-propargyl amide handles
in small-molecule drugs decage succesfully in both aqueous and cell media under
nontoxic quantities of \ce{Pt} salts.
Studies in zebrafish models for the treatment of colorectal cancer showed
promising results.

% TODO:
% SCHEME I from the paper.
% Also, FIGURE I (for the reaction).
% FIGURE II(f) has some info for the calculated mechanism.

As a highlight, it was experimentally found that water was required to generate
the active catalyst, which means activation of metal-chloride bonds by water,
which is rare~\cite{Vidal_2018}.

% TODO:
% THERE ARE KINETIC CONSTANTS IN THE PAPER.\@

Calculations indicate that the reaction takes place through a stepwise process
starting with coordination of the substrate molecule to \ce{Pt(II)}, followed by
an intramolecular attack of the carbonyl oxygen to the pentynoyl moiety,
producing a five-membered ring intermediate.
From this intermediate, different decomposition pathways were explored.
The one with lowest energy encompasses a hydration of the intermediate, which
then further decomposes to liberate the free amine.
A subsequent hydrolysis recovers the metal complex.

% TODO:
% SI MOVIE (LINK) AND FIGURES 20 and 21

The intermediate \ce{CS_0} identified by LC-MS was determined by the computational
methodology.
The difference in activation free energy between both substrates was found to
be around 2.8~\kcalmol.

A new decaging reaction of alkynes with platinum complexes were developed,
aiming at the release secondary amines from stable tertiary amides.
The reaction happens successfully both in cell cultures and living organisms.
We showed that the reaction takes place by a platinum-mediated intramolecular
cyclization mechanism and that water works as the metal-activating agent.
Our computational model matches well the thorough LC-MS characterization of the
reaction intermediates.
The reaction was further extended to work with \ce{N}-propargyl groups with
comparable efficacies to palladium-mediated depropargylation~\cite{Coelho_2019}.

In the group of Prof.~Bernardes, it was adapted for the synthesis of a
noninternalizing ADC (WHAT IS THIS?), resulting in drug release upon
administration of platinum complexes in cancer cells.
It was also demonstrated to work effectively in colorectal cancer zebrafish
xenograft model employing nontoxic amounts of CisPt in order to activate a
prodrug of the anticancer agent 5-FU (WHAT IS THIS AGAIN?), which effectively
led to significant tumor reduction \emph{in vivo}.

This adds a significant decaging strategy for the toolbox of chemical biology
applications.
It reaction is effective in aqueous systems encompassing high concentrations of
salts with high yields and reaction rates, similar to the standard palladium
decaging metal (WHAT?).
A downside is the sucetibility of the reaction towards the presence of
nucleophiles, which leads to one order of magnitude slower rates.
It is also compatible with celullar environments, although the presence of
biomolecules/nucleophiles reduces overall yields considerably.
These results pave the way for future developments of platinum-mediated
decaging reactions.

\section{Paper}

The publication can be read in full next.
\citeauthor{Oliveira_2020}~\cite{Oliveira_2020}
is licensed under a
Creative~Commons~CC-BY~license~\ccby~\cite{ACS_CCBY_2014}.
% TODO: choose where to put it
% Given that this work was featured in the journal's cover,
% it has received some local media coverage~\cite{noticias-da-ufsc2020}.

\includepdf[pages=-]{pubs/oliveira2020-paper2.pdf}
