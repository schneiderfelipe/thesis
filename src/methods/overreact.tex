\section{Automatic determination of kinetics using \overreact}%
\label{sec:overreact-methods}

\overreact is an open-source library, software package and command-line
application for building and analyzing
homogeneous microkinetic models from first-principles
calculations~\cite{Schneider_2022,overreact2021zenodo}.
It propagates chemical reactions over time using only data available from
computational chemistry calculations.

All differential equations are and their parameters are inferred from a
reaction model and calculations provided by the user.
Simultaneous reactions are easily solved, including parallel and concurrent
reactions, pre-equilibration and even constant concentration reactants.

Furthermore, \overreact is able to take most of the relevant physics of the problem into
account in a semi-automated fashion.
This includes concentration effects, symmetries, quantum
tunneling, standard state corrections, implicit and explicit solvation, proper treatment
of energy contributions and dispersion corrections.
In most cases, in particular where solvation effects are weak, results matching
experimental data are obtained.

It is open source\footnote{Code is available at
	\url{https://github.com/geem-lab/overreact-guide}.}, free of charge, available through the Python Package Index (PyPI) and is
distributed under the MIT license.
An online user manual is also
available\footnote{The user guide can be found at \url{https://geem-lab.github.io/overreact-guide/}.}.

USE THE SI FROM THE PAPER TO ADD CONTENT HERE.\@

\section{Thermochemical partition functions}

It is possible to make use of computational calculations to obtain
thermochemical data and, in particular, the thermochemical partition
functions.
This is routinely achieved by standard computational chemistry packages.
This data can be used to estimate the thermodynamic properties of molecules and
whole systems.

Não apenas seus autovalores nos permitem checar se uma otimização alcançou de fato um mínimo local, mas importantes funções de estado termodinâmicas são acessíveis através da matriz hessiana, valendo-se da aproximação QRRHO para gases ideais.

teoria do estado de transição~\cite{TransitionStateTheory}. %(\cref{sec:tst}).

\section{Why predicting chemical reactions is hard}

Perfect prediction of chemical reactions is an unforgiving problem.
Since reaction rate constants depend exponentially on the activation Gibbs' free energy,
small deviations on the latter exponentally increase errors on the latter.
As such, given an activation energy estimate $\Delta G^\ddagger$
on the true value $\Delta \widehat{G}^\ddagger$
with error $\epsilon$,
%
\begin{equation}
	k = \kappa \frac{k_B T}{h} e^\frac{- \Delta G^\ddagger}{R T}
	= \kappa \frac{k_B T}{h} e^\frac{- \left(\Delta \widehat{G}^\ddagger + \epsilon\right)}{R T}
	% = \kappa \frac{k_B T}{h} e^\frac{- \Delta \widehat{G}^\ddagger}{R T}
	% e^\frac{- \epsilon}{R T}
	= \widehat{k} e^\frac{- \epsilon}{R T}
\end{equation}
%
where $k$ and $\widehat{k}$ are the predicted and true chemical reaction constants, respectively.
Thus, at room temperature, an error of 1.36--2.73~\kcalmol gives rise to 10--100$\times$ error
in the reaction rate constant, with larger errors found in lower temperatures.
This is particularly important, as popular DFT methods commonly achieve accuracies of
2--3~\kcalmol for many molecules~\cite{Becke_2014,Bogojeski_2020}.
Moreover, an error of 0.41~\kcalmol produces a twofold error in the constant.
This goes to show that the so called ``quantum chemical accuracy'' of
$<$1~\kcalmol~\cite{Bogojeski_2020}
is not enough for the prediction of chemical reactions on par with experimental results.
It goes without saying that this gives rise not only to a demand
for more precise quantum chemical methods,
but also for methods aiming at mitigating the effect of such errors
in computational predictions of chemical reactions.

One might go one step further and investigate how this error in the reaction rate constant
is related to individual errors in activation enthalpy and entropy errors.
%
\begin{equation}
	k = \kappa \frac{k_B T}{h} e^\frac{- \Delta H^\ddagger}{R T}
	e^\frac{  \Delta S^\ddagger}{R}
	= \kappa \frac{k_B T}{h} e^\frac{- \left(\Delta \widehat{H}^\ddagger + \chi\right)}{R T}
	e^\frac{        \left(\Delta \widehat{S}^\ddagger + \sigma\right)}{R}
	% = \kappa \frac{k_B T}{h} e^\frac{- \Delta \widehat{H}^\ddagger}{R T}
	% e^\frac{  \Delta \widehat{S}^\ddagger}{R}
	% e^\frac{- \chi}{R T}
	% e^\frac{  \sigma}{R}
	= \widehat{k}
	e^\frac{- \chi}{R T}
	e^\frac{  \sigma}{R}
\end{equation}
%
The above suggests that, all things equal,
errors in the predicted activation entropy dominate at high temperatures,
while being less significant
than activation enthalpy errors at low temperatures.
The balance between the two will on the other hand depend on the actual reaction at hand.

\section{``First-principle'' calculations}

Strictly speaking, computational first-principle calculations encompass
methodologies that do not use experimental data either in their development or
application.
Broadly speaking, the first-principle calculations can be referred to
wavefunction or density-functional theory (DFT) calculations.

Os perfis reacionais resultantes são suficientes para a predição de parâmetros
cinéticos e termodinâmicos das reações, que foram comparados aos resultados
disponíveis na literatura. %(\cref{sec:tst}).

\subsection{Quantum tunneling effects}

Hydrogen abstraction reactions (HAA) comprise a class of reactions where
quantum tunneling is often very important~\cite{Bim2018}.
A particular group therein is the homolysis of \ce{C-H} bonds by strong
oxidants, which oftentimes the rate-limiting step in many transformations, and
a particularly key step in the substrate activation by numerous
metalloenzymes~\cite{Bim2018}.

\section{Microkinetic modelling}

Microkinetic modelling is a technique used to predict the outcome of complex
chemical reactions.
It can be used to investigate the catalytic transformations of molecules by
propagating a system of ordinary differential equations modelling the chemical
reactions.

The technique can be made first-principle by making use of pure computational
chemistry predictions.
It is able to take into account effects that sole use of Gibbs' free energies
are not able to, such as concentrations of species and complex time dynamics.

\subsection{Design}

\overreact is a second iteration on an earlier attempt to build a
homogeneous microkinetic analyzer from first-principles
calculations~\cite{pyrrole2019zenodo}.
Some things were learned from that first attempt:
i.\ data is relatively easy to obtain from computational chemistry calculations,
but it is not always in its optimal form;
ii.\ solving differential equations is relatively easy in general, but hard to
make it work for a wide range of problems mostly due to stiffness of the
equations;
iii.\ transforming knowledge about chemical structures into knowledge about the
chemical reactions is hard to be done in an automated fashion;
iv.\ having a good pipeline for data processing makes it easy to add new
features to the system.

\subsection{Automatic differentiation}

Instead of employing numerical differentiation, whose precision depends on the
particular step size, automatic differentiation can be used to produce
analytical derivatives in a precise, efficient and automated way.

One conceptually simple way of doing this is by \emph{forward differentiation}
through dual numbers, where the real values are extended by an infinitesimal
part in a trick that is not far from the concept of imaginary numbers.
The pair of numbers can be added component-wise, and form a commutative algebra
by making use of a simple multiplication rule that follows from the property
$\epsilon^2 = 0$:
\begin{equation}
	(a, b) * (c, d)
	\equiv (a + b\epsilon)(c + d\epsilon)
	= a c + (a d + b c)\epsilon
	\equiv (a c, a d + b c)
\end{equation}
It is not hard to show that, by extending the domain of any real polynomial to
dual numbers, one obtains
\begin{equation}
	P((a, b)) \equiv P(a + b\epsilon) = P(a) + b P'(a) \epsilon
\end{equation}
where $P'(a)$ is the \emph{analytic} derivative of $P(a)$.

\overreact employs a sligthly more complex scheme called
\emph{backward differentiation}, available through Google's JAX library, that
is more efficient for functions of many variables.
