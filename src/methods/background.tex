\section{Theoretical Background}%
\label{sec:background-methods}

Each of the steps in a proposed reaction mechanism is computational modelled
by optimizing the structures of reactants, transition states, intermediates, and products.
While the first two are minima on the potential energy surface, transition
states represent a local maximum on the potential energy surface connecting
reactants and products.
This characterizes transition states as saddle points and, therefore, requires
a different method of optimization than the other structures.
The potential energy surface can be regarded as a function from atomic coordinates to potential energy and,
as such, its existence is a result of the Born-Oppenheimer approximation,
where the electronic energy and wavefunction depends on the nucleus coordinates
only parametrically.
CITE BORN-OPPENHEIMER.\@
In this sense, nuclei are treated classically in this scheme, which is reasonable for most problems of interest in chemistry.
In contrast, other methods exist, where the nuclei are treated quantum mechanically as well,
but they are much more expensive.

In any case, optimization methods require an estimate of energies and gradients
of the potential energy surface.
There are many techniques available to accomplish that,
ranging from simple and low-cost semiempirical methods,
all the way throught highly accurate and expensive multideterminantal wavefunction calculations.
CITE, JACOB'S LADDER, HIERARCHY OF MODELS, ETC.\@
Most of these techniques require self-consistent field approximations, at least initially.

The works presented in this thesis made use of the most popular technique nowadays,
the density functional theory (DFT)
method~\cite{Hohenberg_1964,Kohn_1965,Perdew_2014,Kryachko_2014,Yu_2016} for
that.
A brief discussion of the theory can be found in~\cref{sec:dft}.

\subsection{Density functional theory (DFT)}\label{sec:dft}

This work has made extensive use of the density functional theory (DFT).
As such, a very brief introduction to the theory is provided here.
The interested reader is invited to read the cited references for more
details~\cite{Hohenberg_1964,Kohn_1965}.

The main equation used in the DFT is the following (\cref{eq:KS}):
%
\begin{equation}
	\left(-\frac{1}{2} \laplacian
	+ v_\text{ext}
	+ v_\text{eff}
	\right) \psi_i
	= \epsilon_i \psi_i
	\label{eq:KS}
\end{equation}
%
where $\psi_i$ is $i$th molecular orbital, $v_\text{ext}$ is the external
potential due to the atomic nuclei, $\epsilon_i$ is the electronic energy
of the $i$th orbital and $v_\text{eff}$ is the effective potential for a given
density functional (an approximation of the actual electron-electron
interaction).
This equation is analogous to the Hartree-Fock one~\cite{Szabo_1996}
and is solved self-consistently as well.
$v_\text{eff}$ consists of classical Coulomb interactions, as well as exchange
and correlation approximation
potentials~\cite{Perdew_2014,Kryachko_2014,Yu_2016}.
Other additive terms can appear in the potential as well, such as solvation
approximations when employed~\cite{Marenich_2009,Marenich_2012}.
The various ways of approximating the exchange and correlation potentials give
rise to a number of different density functionals developed in the
literature~\cite{Chai_2008a,Chai_2008b,Goerigk_2011,Arago_2011,Salzner_2011,Burns_2011,Minenkov_2012,DFT2016_poll}.

In the present work, all problems were treated by expanding the molecular
orbitals as a linear combination of basis functions centered at the atomic
nuclei~\cite{Szabo_1996,Helgaker_1997,Jensen_2012,Hill_2012}.
The literature on basis set functions is extensive~\cite{Ditchfield_1971,Hehre_1972,Hariharan_1973,Hariharan_1974,Gordon_1980,Francl_1982,Clark_1983,Frisch_1984,Binning_1990,Blaudeau_1997,Rassolov_1998,Rassolov_2001}.
We invite the interested reader to the references on each work for more about
the levels of theory used in each case.

IMPACTS OF DFT APPROXIMATIONS ON THE PES?\@

% TODO: check that citations follow the short compact convention e.g. [18-25]

\subsection{Single-molecule property predictions}\label{sec:optimizations}

\cref{eq:KS} allows us to obtain energy and gradient estimates (required for
the prediction of geometrical, thermochemical and kinetic properties) with a
reasonably good cost-effectiveness.
WHY, HOW, SUPPORT WITH WHAT?\@ CITATIONS!\@

The most basic property required for any computational study is the molecular
geometry.
It is obtained by minimizing the energy with respect to the atomic positions.
Commonly used and powerful family of optimization methods employed in the
literature is the quasi-Newton method, where energies and gradients are
computed, while an approximation to the second-derivative matrix (also known as
a Hessian matrix) is updated at each step~\cite{Banerjee_1985,Schlegel_1987}.
CRUCIAL EQUATIONS SHOULD BE HIGHLIGHTED!\@

An approximation to the second-derivative matrix is employed in the algorithm
for efficiency, but the actual matrix can be computed after the minimization is
complete.
This matrix provides important information about the region in the potential
energy surface where the energy is minimum as we will see in the following.
EVERYTHING IS MISSING, EITHER REMOVE STUFF OR SHOW STUFF.\@
ALSO, CITATIONS ARE LACKING!\@
DON'T APPROACH THIS TANGENTIALLY.\@

\subsubsection{Transition state optimizations}\label{sec:ts-optimizations}
% TODO: the logical sequence is talk about ground-state optimizations first.
% Then move to transition state optimizations.

A structure to be a minimum on the potential energy surface means that any
displacement of the structure will result in an increase in the potential
energy.
As a consequence, its second-derivative matrix
is positive semi-definite (all eigenvalues are non-negative).
On the other hand, first-order saddle points are characterized by behaving as
a minimum with respect to all displacements of the structure except in a single
direction.
Displacement in this particular direction leads to a decrease in the potential
energy and is associated with a single negative eigenvalue of the
second-derivative matrix.
ADD A GRAPHICAL SCHEME TO SHOW THIS!\@
By calculating the second-derivative matrix, we can determine whether a structure
is a minimum, saddle point or otherwise.

Obtaining transition states can be done by a number of methods.
For instance, eigenvector following (\emph{EF})
method~\cite{Banerjee_1985,Schlegel_1987,Mauro_2005}
attempts to maximize the energy along a given promising direction, while
minimizing the energy along all other directions.
Another very much used is the \emph{STQN} method, developed
by~Peng~and~Schlegel~\cite{Peng_1993,Peng_1996},
It consists in obtaining an estimate of the transition state from the known
structures of reactants and products (QST2).
An associated method allows the use of a custom guess for the transition state
(QST3).
% TODO: one has to list the main characteristics of each method, advantages and disadvantages, etc.
% It is nice to have equations and/or schemes to highlight the most important things and is mandatory.
The method is useful since
i.\ knowledge of the transition state is optional,
ii.\ it avoids calculating the second-derivative matrix for each step.
The nudge-elastic band method is in the same vein as the \emph{STQN} method,
but attempts to minimize a discretized trajectory connecting reactants and
products along the transition state, and can thus make use of information about
this curve.

\subsection{Prediction of kinetics and thermodynamics for chemical reactions}

\subsubsection{Transition state theory}\label{sec:tst}

Transition state theory
is based on the idea that the reactant ground state complex is in equilibrium
with the transition state structure~\cite{TransitionStateTheory}.
It is summarized by the Eyring equation.

HAS TO BE REWRITTEN, BUT ONLY THE ESSENTIALS.\@

\subsection{When things go wrong:
	bulk effects}

Here are some situations where predictions become poor due to the fact that our
calculations are too small.

FLOATING PARAGRAPH!\@

\subsubsection{Determining \emph{pKa}s}\label{sec:pka}

WHY?\@
THIS MAKES THE TEXT NOT FOLLOW A CONCATENATED CHAIN OF THOUGHT.\@

In the computational prediction of \emph{pKa} values,
the direct dissociation approach (\cref{eq:pKa-equilibrium})
leads to problems in the evaluation of the solvated proton,
first because it is impossible to do electronic calculations for a zero-electron
system~\cite{Ding_2009,Sumon_2012}.
%
\begin{equation}
	\ce{HB (aq) <=>[$\emph{pKa} (\ce{HB})$] B^- (aq) + H^+ (aq)}\text{.}
	\label{eq:pKa-equilibrium}
\end{equation}
%
Second, due to the covalent nature of the interaction between the \ce{H^+} ion
and the surrounding environment, which leads to the formation of clusters such
as \ce{H3O+}, \ce{H5O2+}, etc.~\cite{Sumon_2012}.

One can employ the experimental proton solvation free energy
($-$1104,5~$\pm$~0,3~kJ/mol~\cite{Tissandier_1998,Marenich_2009}) in a
semi-empirical fashion, but the trustworthiness of the results is still highly
debated in the literature~\cite{Yang_2013}.
In fact, this often produces \emph{pKa} values that are far from the expected
for simple carboxylic acids, for instance.
Errors in the order of 7 \emph{pKa} units can be expected with this approach,
as can be found in the literature~\cite{Pliego_2002,Ding_2009}.
The major source of error can be attributed to the lack of explicit
solute-solvent interactions~\cite{Pliego_2002}.

A better approach is to use a relative determination method~\cite{Ding_2009}.
This avoids the problems associated with the direct dissociation approach
by calculating the associated problem of proton exchange with a known reference
acid, such as acetic acid~\cite{Goldberg_2002}:
%
\begin{equation}
	\ce{HA (aq) + B^- (aq) <=>[$\Delta{} G$] HB (aq) + A^- (aq)}\text{.}
	\label{eq:pKa-indireto}
\end{equation}
%
Since both sides of the~\cref{eq:pKa-indireto}
have the same net charge and, hopefully both acid and conjugated base are, in
general, of similar structure and size, favorable error cancelation is expected
in the Gibbs free energy ($\Delta G$).
The $\emph{pKa}(\ce{HA})$
can thus be written as a function of $\emph{pKa}(\ce{HB}, \text{exp.})$:
%
\begin{equation}
	\emph{pKa}(\ce{HA})~=~\emph{pKa}(\ce{HB}, \text{exp.}) + \frac{\Delta G}{\ln(10) R T}\text{.}
\end{equation}
%
It is customary that the error obtained by this method to be less than 1 \emph{pKa}
unit for many solutes and solvents~\cite{Ding_2009}.
