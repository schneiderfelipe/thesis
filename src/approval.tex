% Isto é um exemplo de Folha de aprovação,
% elemento obrigatório da NBR
% 14724/2011 (seção 4.2.1.3). Você pode utilizar este modelo até a aprovação
% do trabalho. Após isso,
% substitua todo o conteúdo deste arquivo por uma
% imagem da página assinada pela banca com o comando abaixo:
%
% \begin{folhadeaprovacao}
% \includepdf{folhadeaprovacao_final.pdf}
% \end{folhadeaprovacao}
%
\begin{folhadeaprovacao}

	\begin{center}
		{\ABNTEXchapterfont\large\imprimirautor}

		\vspace*{\fill}\vspace*{\fill}
		\begin{center}
			\ABNTEXchapterfont\bfseries\Large\imprimirtitulo{}
		\end{center}
		\vspace*{\fill}

		% \hspace{.45\textwidth}
		% \begin{minipage}{.5\textwidth}
		% 	\imprimirpreambulo{}
		% \end{minipage}%
		% \vspace*{\fill}
	\end{center}

	O presente trabalho em nível de Doutorado foi avaliado e aprovado, em \imprimirdata, pela banca examinadora composta pelos seguintes membros:

	\begin{center}
		Prof.\ Dr.\ Maximiliano Segala \\ IQ/UFRGS % Relator

		Prof.\ Dr.\ Amir A. M. de Oliveira Jr. \\ EMC/UFSC

		Prof.\ Dr.\ Ataualpa A. C. Braga \\ IQ/USP

		Prof.\ Dr.\ Luciano N. Vidal \\ DAQBI/UFTPR
	\end{center}

	Certificamos que esta é a versão original e final do trabalho de conclusão que foi julgado adequado parra obtenção do título de Doutor em Química, na área de concentração em Físico-Química.

	\assinatura{\textbf{Prof.\ Dr.\ Daniel Lázaro G.\ Borges} \\ Coordenação do Programa~de~Pós-Graduação}
	\assinatura{\textbf{\imprimirorientador} \\ Orientador}

	\begin{center}
		\vspace*{0.5cm}
		{\large\imprimirlocal}
		\par
		{\large\imprimirdata}
		\vspace*{0.5cm}
	\end{center}

\end{folhadeaprovacao}
