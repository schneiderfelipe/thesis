\chapter{Practical computational chemistry for the investigation of reaction
  mechanisms and transition state searches}%
\label{ch:tutorial}

This appendix is an account of techniques used and hard tested during my~PhD.
Most of them are of practical value and might be useful for other researchers
to use.

\section{General remarks about transition state structures}

Obtaining transition state structures is not an easy task, at least not as easy
as optimizing structures to minima.
One core reason is that, although we have a good understanding about how
minimum energy structures look like, it is not so easy with transition states,
as they necessarily are in \emph{between} observed minima (in the sense of
potential energy surface).

One could imagine that bond lengths, angles and dihedrals be exactly at the
middle between reactants and products, but reality is more complicated than
that.
As suggested by Hammond's postulate (and more generally the Bell-Evans-Polanyi
principle), a transition state will more closely resemble the reactant
structure if the reaction is exothermic (early transition state), and more the
product if endothermic (late transition state).

CITE THE SECTION ABOUT BEP AND HAMMOND.\@

As such, it is reasonable to assume that the transition state structure can
only be exactly at the middle if the reaction is thermoneutral, i.e., reactants
and products sharing the same energy.
This occurs, for example, in the umbrella effect of amines, or in enantiomeric
interconversions.

\subsection{Using Hammond's postulate for estimating transition state
	structures}

RELEVANT PAPERS FOR THE DISCUSSION ARE in~\citeauthor{Cremer_2012}~\cite{Cremer_2012}
AND CITATIONS 4--6 THEREIN.\@

The rate at which the reaction complex changes in the start of a reaction, as
it walks along the reaction path, is related to the curvature of energy in the
reaction valley along this path.
It is worth mentioning that, since reactions tend to go along the least energy
path, this curvature is most likely to be wide (i.e., less steep) when compared
with the other directions in the reaction valley.
The reason is that molecular vibrations that don't led to the reaction are
normally stiffer than the reaction direction.

Along the lines of Hammond's postulate and the Bell-Evans-Polanyi principle,
there is evidence (REF 14 OF THE PAPER ABOVE) that the reaction energy is
proportional to the difference in position between the transition state and the
exact midpoint of the reaction path.
Although this requires knowledge of the reaction path and its midpoint
position, and approximation for the linear path with a single changing degree
of freedom is useful for guessing the transition state:
%
\begin{equation}
	\Delta E^\circ \propto \Delta s
	= s_\text{TS} - s_\text{midpoint}
	\approx d_\text{TS} - \frac{d_\text{P} + d_\text{R}}{2}
\end{equation}
%
where $s$ stands for the actual distance in some respect (e.g., root mean
squared deviation), $d$ is the distance in a single degree of freedom, and R,
TS, and P correspond the reactant, transition state and product, respectively.
We are also setting $d_\text{midpoint}$
to~$\frac{d_\text{P} + d_\text{R}}{2}$.
As such, using a linear approximation,
%
\begin{equation}
	d_\text{TS}
	\approx d_\text{midpoint} + \frac{\Delta E^\circ - b}{a}
\end{equation}
%
where $a$ and $b$ are constants to be determined.
By making the reasonable assumption that the transition state should be at the
midpoint for thermoneutral reactions, we learn that $b$ is zero.
Furthermore, if we consider $\Delta d < 0$ to be an early transition state, we
determine that $a$ is positive, since the reaction is exothermic
($\Delta E^\circ < 0$) in this case due to the BEP principle.
A typical value for $|a|$ is around
$1.5 \times 10^{-2}$
$\sqrt{\text{amu}}\cdot\text{Bohr} / \text{\kcalmol}$
(JUSTIFY???).

Partial evidence for this can be found in the literature on hydrogen
abstraction reactions (HAA)~\cite{Bim2018}.
\citeauthor{Bim2018} suggest a relationship between Marcus' reorganization
energy $\lambda$ and an asynchronicity parameter $\nu$ that can empirically
rationalize the concertedness of HAA reactions as whether the proton or the
electron is transferred first.
The $\nu$ factor is, conceptually, a decomposition between the effective redox
and effective acidobasic contributions to the reaction.
A perfectly synchronous HAA reaction would have $\nu = 0$, whereas positive
values favor electron transfer (ET) and negative values favor proton transfer
(PT).
They showed that $\lambda$ reaches its maximum where $|\nu|$ is lowest.
This implies that, for this particular class of reactions, among systems with
similar reaction free energies, the highest barrier is to be expected for the
most synchronous proton-coupled electron transfer~\cite{Bim2018}.

\section{Some methods for obtaining transition states}

The general idea is that the technique will depend on the information you have.
If you have
\begin{enumerate}
	\item only reasonable reactants or only reasonable products, start with conformational search,
	      metadynamics and scans. Chemical knowledge can help in finding candidates,
	      though.
	\item both reasonable reactants and products, but no idea about the transition state, you can use
	      all of the above, plus QST2 and NEB.\@ Chemical knowledge plus simple estimates
	      above can help on finding a transition state estimate, though.
	\item reasonable reactants and products, but crude transition state
	      estimate, use QST3,\ NEB, and the above. The
	      goal here is validate your transition state structure candidate as a
	      \emph{true} transition state.
	\item reasonable transition state, use an eigenvector following method. Good ones
	      provide normal mode projection so that you can better control de optimization
	      direction. Furthermore, a good initial hessian is also nice to have.
\end{enumerate}

If have structures for both reactants and products, you can use techniques such
as quadratic synchronous transit (QST2) or nudge-elastic band (NEB) to obtain a
good guess on the transition state.

In such calculations, it is normally mandatory that the atom numbering match
between atoms in both reactants and products.

In case you already have a reasonable guess for the transition state, both
methods above (QST3 and NEB) can take advantage of that.

\subsection{What if this don't work?}

Sometimes you end up with an unconverged calculation.
In other words, the calculation has performed some cycles, but then crashed.
Changes are the final state of the calculation already has a better guess for
the transition state.
This structure can be used for further calculations.

\section{What to do once you found a transition state candidate}

Once you got a transition state candidate, both with its structure optimized
and calculated frequencies, you should check that:
%
\begin{itemize}
	\item only a single frequency appears, and
	\item that it corresponds to the expected bond breaking and forming.
\end{itemize}

\subsection{If you are finding a set of homologous transition states}

Focus on finding the simplest one first, then use it as a template for the
other ones.
