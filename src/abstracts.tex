\setlength{\absparsep}{18pt} % ajusta o espaçamento dos parágrafos do resumo

% resumo em inglês
\begin{resumo}[Abstract]
	This thesis aims to explore the application of computational chemistry to the elucidation of complex chemical~reaction mechanisms.
	Two computational-experimental investigations using density functional theory calculations to understand chemical~reaction mechanisms are presented in the thesis,
	as is a publication on the development of a software package for simulating complex chemical~reaction networks.
	The first publication consists of an experimental-computational hybrid investigation on the \ce{Pd(II)}-mediated bioorthogonal uncaging of propargyl-protected hydroxyl groups for the purpose of activating prodrugs
	(\fullcite{Coelho_2019}).
	In the second publication,
	a joint computational-experimental collaboration as well,
	we investigated the \ce{Pt(II)}-mediated depropargylation of pentynoyl amide and~\ce{N}-propargyl handles for drug-activation delivery in cancer therapy
	(\fullcite{Oliveira_2020}).
	Quantum mechanical calculations were key to support the proposed chemical~reaction pathways in both cases.
	In the third publication
	we describe the development of~\overreact{},
	a user-friendly,
	open-source program that can be used
	to automatically perform microkinetic modelling of complex chemical~reactions in solution or gas-phase
	using data from first-principles quantum chemical calculations
	(\fullcite{Schneider_2022}).
	Limitations of the findings are discussed,
	and potential future work is suggested.

	\vspace{\onelineskip}

	\noindent
	\textbf{Keywords}:
	catalysis.\ computational chemistry.\ chemical~reaction mechanisms.\ microkinetics.\ Python.\ DFT.\@
\end{resumo}

% resumo em português
\begin{resumo}[Resumo]
	\begin{otherlanguage*}{brazil}
		Esta tese visa explorar a aplicação da química computacional na elucidação de mecanismos químicos complexos.
		São apresentadas duas investigações computacional-experimentais usando cálculos de teoria da função densidade para compreender mecanismos de reações químicas,
		assim como uma publicação sobre o desenvolvimento de um pacote de \emph{software} para simular redes de reações químicas complexas.
		A primeira publicação consiste em uma investigação híbrida experimental-computacional sobre o descascamento bioortogonal de grupos hidroxila protegidos por propargila mediado por \ce{Pd(II)} com o objetivo de ativar pró-drogas
		(\fullcite{Coelho_2019}).
		Na segunda publicação,
		uma colaboração computacional-experimental também,
		investigamos a despropargilação mediada por \ce{Pt(II)} de amida de pentinoíla e alças de \ce{N}-propargila para entrega e ativação de medicamentos em terapia contra câncer
		(\fullcite{Oliveira_2020}).
		Cálculos mecânicos quânticos foram cruciais para apoiar os caminhos de reação propostos em ambos os casos.
		Na terceira publicação,
		descrevemos o desenvolvimento do~\overreact{},
		um programa de código aberto e fácil de usar que pode ser usado
		para realizar modelagem microcinética de reações químicas complexas em solução ou fase gasosa
		de maneira automática e
		usando dados de cálculos químicos quânticos de primeiros princípios
		(\fullcite{Schneider_2022}).
		As limitações dos resultados são discutidas e são sugeridos trabalhos futuros potenciais.

		\vspace{\onelineskip}

		\noindent
		\textbf{Palavras-chave}:
		catálise.\ química computacional.\ mecanismos de reação química.\ microcinética.\ Python.\ DFT.\@
	\end{otherlanguage*}
\end{resumo}

