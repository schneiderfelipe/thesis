\setlength{\absparsep}{18pt} % ajusta o espaçamento dos parágrafos do resumo

% resumo em português
\begin{resumo}[Resumo]
	\begin{otherlanguage*}{brazil}
		Esta tese tem como objetivo explorar a aplicação da química computacional na elucidação de mecanismos de reações químicas complexas.
		Duas investigações computacionais-experimentais usando cálculos de teoria do funcional de densidade para entender mecanismos de reações químicas são apresentadas na tese,
		bem como uma publicação sobre o desenvolvimento de um pacote de \emph{software} para simular redes complexas de reações químicas.
		A primeira publicação consiste em uma investigação híbrida experimental-computacional sobre o desprotegimento bioortogonal mediado por \ce{Pd(II)} de grupos hidroxila protegidos por propargila para o propósito de ativação de pró-fármacos
		(\fullcite{Coelho_2019}).
		Na segunda publicação,
		também uma colaboração computacional-experimental conjunta,
		investigamos a despropargilação mediada por \ce{Pt(II)} de amida de pentinoíla e \ce{N}-propargila para liberação e ativação de fármacos na terapia do câncer
		(\fullcite{Oliveira_2020}).
		Cálculos de mecânica quântica foram fundamentais para apoiar os mecanismos de reação química propostos em ambos os casos.
		Na terceira publicação,
		descrevemos o desenvolvimento do~\overreact{},
		um programa de código aberto e fácil de usar que pode ser usado
		para realizar automaticamente modelagem microcinética de reações químicas complexas em solução ou fase gasosa
		usando dados de cálculos químicos quânticos de primeiros princípios
		(\fullcite{Schneider_2022}).
		As limitações dos resultados são discutidas,
		e trabalhos futuros potenciais são sugeridos.

		\vspace{\onelineskip}

		\noindent
		\textbf{Palavras-chave}:
		catálise.\ química computacional.\ mecanismos de reação química.\ microcinética.\ Python.\ DFT.\@
	\end{otherlanguage*}
\end{resumo}

% resumo expandido em português (TODO)
\begin{resumo}[Resumo Expandido]
	\begin{otherlanguage*}{brazil}
		\textbf{Introdução}\linebreak
		Esta tese tem como objetivo explorar a aplicação da química computacional para a elucidação de mecanismos de reações químicas complexos,
		utilizando cálculos de teoria do funcional da densidade (DFT) e modelagem microcinética.
		Ao longo do trabalho,
		são apresentadas duas investigações computacional-experimentais,
		bem como o desenvolvimento de um pacote de \emph{software} para simular redes de reações químicas complexas.

		\textbf{Objetivos}\linebreak
		O principal objetivo desta tese é investigar e desenvolver métodos computacionais para entender mecanismos de reações químicas complexas,
		utilizando cálculos de DFT e modelagem microcinética.
		Além disso,
		a tese busca contribuir com o desenvolvimento de um pacote de \emph{software} para simular redes de reações químicas complexas,
		facilitando a aplicação dos métodos computacionais desenvolvidos.

		\textbf{Metodologia}\linebreak
		No primeiro capítulo,
		é realizada uma introdução aos conceitos fundamentais de química computacional,
		teoria do funcional da densidade e microcinética de reação química.
		Além disso,
		são apresentadas as principais técnicas e métodos empregados ao longo do desenvolvimento da tese.

		No segundo capítulo,
		são introduzidas as contribuições científicas desta tese,
		destacando os principais resultados e o impacto das publicações geradas ao longo do trabalho.
		São apresentadas tanto as contribuições menores,
		quanto as principais contribuições em cada um dos estudos realizados.

		\mbox{\textbf{Resultados~e~Discussão}}\linebreak
		Na primeira investigação,
		apresentada no Capítulo 3 (\fullcite{Coelho_2019}),
		foi realizado um estudo híbrido experimental-computacional sobre o descascamento bioortogonal de grupos hidroxila protegidos por propargila mediado por \ce{Pd(II)} com o objetivo de ativar pró-drogas.
		Os cálculos mecânicos quânticos foram fundamentais para apoiar os mecanismos de reação propostos,
		oferecendo uma visão detalhada das etapas envolvidas no processo e contribuindo para o entendimento geral da reação.

		Na segunda investigação,
		apresentada no Capítulo 4 (\fullcite{Oliveira_2020}),
		foi realizada uma colaboração computacional-experimental com foco na despropargilação mediada por \ce{Pt(II)} de amida de pentinoíla e alças de \ce{N}-propargila para entrega e ativação de medicamentos em terapia contra câncer.
		Mais uma vez,
		os cálculos mecânicos quânticos desempenharam um papel crucial na elucidação dos caminhos de reação e na compreensão dos resultados experimentais obtidos.

		No Capítulo 5 (\fullcite{Schneider_2022}),
		é descrito o desenvolvimento do pacote de \emph{software} \overreact{},
		um programa de código aberto e fácil de usar que pode ser utilizado para realizar modelagem microcinética de reações químicas complexas em solução ou fase gasosa de maneira automática,
		utilizando dados de cálculos químicos quânticos de primeiros princípios.
		O \emph{software} é aplicado em diversos estudos de caso,
		demonstrando sua versatilidade e utilidade para pesquisadores da área.

		\mbox{\textbf{Considerações~Finais}}\linebreak
		Por fim,
		no sexto capítulo,
		são discutidas as conclusões,
		limitações dos resultados obtidos e sugeridos trabalhos futuros potenciais.
		Destaca-se a importância dos estudos realizados e o impacto dos resultados obtidos na compreensão dos mecanismos de reação química e no desenvolvimento de novas estratégias para ativação de pró-drogas e terapias contra câncer.
		Além disso,
		são apontadas possíveis direções para pesquisas futuras,
		tanto na área de química computacional quanto na aplicação prática dos resultados obtidos.

		\vspace{\onelineskip}

		\noindent
		\textbf{Palavras-chave}:
		catálise.\ química computacional.\ mecanismos de reação química.\ microcinética.\ Python.\ DFT.\@
	\end{otherlanguage*}
\end{resumo}

% resumo em inglês
\begin{resumo}[Abstract]
	This thesis aims to explore the application of computational chemistry to the elucidation of complex chemical~reaction mechanisms.
	Two computational-experimental investigations using density functional theory calculations to understand chemical~reaction mechanisms are presented in the thesis,
	as is a publication on the development of a software package for simulating complex chemical~reaction networks.
	The first publication consists of an experimental-computational hybrid investigation on the \ce{Pd(II)}-mediated bioorthogonal uncaging of propargyl-protected hydroxyl groups for the purpose of activating prodrugs
	(\fullcite{Coelho_2019}).
	In the second publication,
	a joint computational-experimental collaboration as well,
	we investigated the \ce{Pt(II)}-mediated depropargylation of pentynoyl amide and~\ce{N}-propargyl handles for drug-activation delivery in cancer therapy
	(\fullcite{Oliveira_2020}).
	Quantum mechanical calculations were key to support the proposed chemical~reaction pathways in both cases.
	In the third publication
	we describe the development of~\overreact{},
	a user-friendly,
	open-source program that can be used
	to automatically perform microkinetic modelling of complex chemical~reactions in solution or gas-phase
	using data from first-principles quantum chemical calculations
	(\fullcite{Schneider_2022}).
	Limitations of the findings are discussed,
	and potential future work is suggested.

	\vspace{\onelineskip}

	\noindent
	\textbf{Keywords}:
	catalysis.\ computational chemistry.\ chemical~reaction mechanisms.\ microkinetics.\ Python.\ DFT.\@
\end{resumo}

