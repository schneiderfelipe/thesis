\chapter{Literature review}%
\label{ch:lit-review}

In the past,
computational tools were mainly used to explain chemical reactions after experiments had already been done.
However,
with the recent development of software and hardware technologies,
these tools have become more reliable and can now be used to provide valuable insights and even make predictions~\cite{Ahn_2019}.

Modern technology has made it easier and more affordable to access high-performance computing hardware and efficient quantum chemical modeling software.
This has allowed researchers to use quantum chemical reaction modeling techniques more widely in chemical research,
especially in the field of organic and organometallic catalysis.

Initially,
computer models were used to explain experimental observations,
but with the development of density functional theory (DFT),
it became possible to model molecules of realistic size and complexity with little to no structural simplifications.
This has enabled researchers to use these models to optimize and even predict new reactions that would not be possible with purely experimental approaches.

\citeauthor{Ahn_2019} highlighted several studies that used computational predictions to explain unexpected reactivities or provide mechanistic insights for organic and organometallic reactions,
which led to improved experimental results.
According to the authors,
the key to these successful applications is combining theory and experiment,
which allows for the combination of empirical knowledge with precise computed values\cite{Ahn_2019}.

Computational models can be used to virtually optimize novel or practical catalysts and develop new reactions before they are tested in the lab.
This can save time and effort compared to experimenting.
Virtual screening is becoming more popular as a way to find molecules and materials with desirable properties.
However,
automated optimization techniques are not always successful and often require classical chemical logic and conventional strategies of reaction design.
Density Functional Theory (DFT) calculations can provide energies and structures of intermediates that may not be accessible experimentally,
and can also allow for a qualitative examination of the electronic structure that is responsible for the numerical result.
This can help to develop general concepts that can be applied to a whole class of reactions.

% TODO: my text below

For centuries,
chemists have seeked to understand the underpinnings of chemical reactions,
and the mechanisms by which they proceed~\cite{Armstrong_1887}.
It is crucial to understand the mechanisms of reaction in order to maximize the efficiency of a reaction
and to minimize unwanted side reactions.
Mechanistic insights can help us predict and control the course of a reaction,
allowing us to tailor a reaction to a particular end goal.
By better understanding reaction mechanisms,
we can design safer,
more efficient processes,
and more powerful synthetic tools.

In order to accomplish this,
many concepts and techniques were devised.
Reactivity descriptors,
such as
Hammond's postulate~\cite{Hammond_1955,Cremer_2012,HammondPrinciple} or
Woodward-Hoffmann rules~\cite{Havinga_1961,Woodward_1965,Dewar_1966,Zimmerman_1966,Woodward_1969,Nobel_1981},
aid in predicting the course of a reaction,
as does frontier orbital theory~\cite{Fukui_1952,Brown_2013}.

All these descriptors have in common a theoretical ground.
With advances in computing technology,
however,
the power of the computational chemistry has become more and more apparent in recent years.
The use of computational techniques allows the accurate prediction of reaction mechanisms,
as well as detailed insight into potential reaction pathways;
possibly previously unknown synthetic pathways can be successfully explored.
In addition,
the usefulness of computational chemistry extends from organic and inorganic synthesis to biochemistry,
allowing the prediction of reaction mechanisms even in complex systems~\cite{Klippenstein_2014}.
By utilizing these powerful computational techniques,
we can better understand the mechanisms of reaction in the laboratory,
and allow us to come up with targeted,
effective reactions.

% TODO: below was computer generated.
Although reactive trajectories are certainly more informative than single product predictions,
CONTINUE (CITE).

However,
a more successful and rigurous way of deriving rational principles that govern
chemical reactions is to employ statistical screening based on first principles
of quantum mechanics (CITE).
% ellided
A second direction of research often involves creation and screening of
tall sets of different small candidate structures,
employing a stark number of calculations with first-principles software.
% ellided
This research develops suitable computational techniques
to both allow screening of large numbers of candidate structures as well as
an in-silico method of support construction supervised by first principles.
Note that the evaluation of support effects in such a general way
has not been possible experimentally,
suggesting that such a methodology may benefit the field of catalysis greatly,
allowing faster screening and evaluation of
different candidate supports.

Based on the underlying structure of most of the natural world,
an interaction graph is a common way of relating
nearest-neighbours for most reactions using machine learning
for the classification of their topology (CITE).

% TODO: computer generated but too simple
Given the experimental difficulties in elucidating the complex pathways and reaction conditions around each elementary kinetics step of a reaction network,
theoretical predictions are needed to precise estimate reaction rates.
This can allow an insightful understanding of the reaction network and would lead to a more efficient and safer design (predicting processes with unwanted by-products),
and a better guidance for future experimental efforts in this area.

% TODO: extra thing to add at the end
Computer modeling of chemical reactions is becoming more and more popular among researchers,
and it is expected that its usefulness in predicting reactions will continue to grow in the near future~\cite{Ahn_2019}.

\citeauthor{Ahn_2019} highlight how computational studies can be used to design new chemical reactions,
by performing calculations that can be used to predict new reactions,
which is different from their traditional role of just explaining what has already been observed in experiments~\cite{Ahn_2019}.
They emphasize, however,
that it is not yet possible to routinely or automatically design new chemical reactions using computers~\cite{Ahn_2019}.
We hope this changes soon.
