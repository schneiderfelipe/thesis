\chapter{Literature review}%
\label{ch:lit-review}

For centuries, chemists have seeked to understand the underpinnings of chemical reactions,
and the mechanisms by which they proceed~\cite{Armstrong_1887}.
It is crucial to understand the mechanisms of reaction in order to maximize the efficiency of a reaction
and to minimize unwanted side reactions.
Mechanistic insights can help us predict and control the course of a reaction, allowing us to tailor a reaction to a particular end goal.
By better understanding reaction mechanisms, we can design safer, more efficient processes, and more powerful synthetic tools.

In order to accomplish this, many concepts and techniques were devised.
Reactivity descriptors, such as
Hammond's postulate~\cite{Hammond_1955,Cremer_2012,HammondPrinciple} or
Woodward-Hoffmann rules~\cite{Havinga_1961,Woodward_1965,Dewar_1966,Zimmerman_1966,Woodward_1969,Nobel_1981},
aid in predicting the course of a reaction, as does frontier orbital theory~\cite{Fukui_1952,Brown_2013}.

All these descriptors have in common a theoretical ground.
With advances in computing technology, however, the power of the computational chemistry has become more and more apparent in recent years.
The use of computational techniques allows the accurate prediction of reaction mechanisms,
as well as detailed insight into potential reaction pathways;
possibly previously unknown synthetic pathways can be successfully explored.
In addition, the usefulness of computational chemistry extends from organic and inorganic synthesis to biochemistry,
allowing the prediction of reaction mechanisms even in complex systems~\cite{Klippenstein_2014}.
By utilizing these powerful computational techniques, we can better understand the mechanisms of reaction in the laboratory,
and allow us to come up with targeted, effective reactions.
