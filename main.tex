\documentclass[
    % # memoir class options
    12pt,
openright,             % chapters start at odd page numbers
    twoside,               % print on both sides (change to `oneside` to
% leave back blank)
    a4paper,
    % # abntex2 class options
%chapter=TITLE,        % uppercase chapter titles
%section=TITLE,        % uppercase section titles
%subsection=TITLE,     % uppercase subsection titles
    %subsubsection=TITLE,  % uppercase subsubsection titles
    % # babel package options
french,            % additional hyphenization
ngerman,           % additional hyphenization
spanish,           % additional hyphenization
brazil,            % additional hyphenization
british,           % main language of the document
    ]{abntex2}

% # Basic packages
\usepackage{lmodern}           % Latin Modern font
\usepackage[T1]{fontenc}       % font encoding
\usepackage[utf8]{inputenc}    % document encoding (e.g., required for diacritics)
\usepackage{indentfirst}       % indent the first paragraph of each section
\usepackage{color}             % color control
\usepackage{graphicx}          % include graphics
\usepackage{microtype}         % typography enhancements

% # Utility packages
\usepackage{import}  % nestable, locally-scoped import directives (note that it
                     % is best to use commands (e.g., `subimport{...}{...}` in
                     % their own lines)
\usepackage{lipsum}  % TODO: remove this and its usage

% # Citations
\usepackage[british,hyperpageref]{backref}   % show where works where cited
\usepackage[alf]{abntex2cite}                % cite according to ABNT

% # Preamble
\subimport{./src/preamble}{config}   % configuration of packages
\subimport{./src/preamble}{spacing}  % spacing and placement of elements
\subimport{./src/preamble}{quadros}  % definition of schemes ("quadros")
\subimport{./src/preamble}{info}     % general information such as title, author, etc.

\makeindex  % compile our index

\begin{document}

\subimport{./src}{before}

\textual{}  % textual elements from now on (i.e., the actual work)

\subimport{./src}{introduction}
\subimport{./src}{literature}
\subimport{./src}{methodology}
\subimport{./src}{results}
\subimport{./src}{discussion}

% ----------------------------------------------------------
% PARTE
% ----------------------------------------------------------
\part{Preparação da pesquisa}
% ----------------------------------------------------------

\subimport{./src}{commands}
% ---

\chapter{Conteúdos específicos do modelo de trabalho acadêmico}%
\label{cap_trabalho_academico}

\section{Quadros}

Este modelo vem com o ambiente \texttt{quadro} e impressão de Lista de quadros
configurados por padrão. Verifique um exemplo de utilização:

\begin{quadro}[htb]
\caption{%
    \label{quadro_exemplo}Exemplo de quadro}
\begin{tabular}{cccc}
    \toprule
    \textbf{Pessoa} & \textbf{Idade} & \textbf{Peso} & \textbf{Altura} \\
    \midrule
    Marcos          & 26             & 68            & 178             \\
    Ivone           & 22             & 57            & 162             \\
    \ldots          & \ldots         & \ldots        & \ldots          \\
    Sueli           & 40             & 65            & 153             \\
    \bottomrule
\end{tabular}
\fonte{Autor.}
\end{quadro}

Este parágrafo incrível apresenta como referenciar o quadro no texto, requisito
obrigatório da ABNT.\@
Primeira opção, utilizando \texttt{autoref}: Ver o~\autoref{quadro_exemplo}.
Segunda opção, utilizando \texttt{ref}: Ver o Quadro~\ref{quadro_exemplo}.

% ----------------------------------------------------------
% PARTE
% ----------------------------------------------------------
\part{Referenciais teóricos}
% ----------------------------------------------------------

% ---
% Capitulo de revisão de literatura
% ---
\chapter{Lorem ipsum dolor sit amet}
% ---

% ---
\section{Aliquam vestibulum fringilla lorem}
% ---

\lipsum[1]

\lipsum[2]

% ----------------------------------------------------------
% PARTE
% ----------------------------------------------------------
\part{Resultados}
% ----------------------------------------------------------

% ---
% primeiro capitulo de Resultados
% ---
\chapter{Lectus lobortis condimentum}
% ---

% ---
\section{Vestibulum ante ipsum primis in faucibus orci luctus et ultrices
posuere cubilia Curae}
% ---

\lipsum[21]

% ---
% segundo capitulo de Resultados
% ---
\chapter{Nam sed tellus sit amet lectus urna ullamcorper tristique interdum
elementum}
% ---

% ---
\section{Pellentesque sit amet pede ac sem eleifend consectetuer}
% ---

\lipsum[24]

% ----------------------------------------------------------
% Finaliza a parte no bookmark do PDF
% para que se inicie o bookmark na raiz
% e adiciona espaço de parte no Sumário
% ----------------------------------------------------------
\phantompart{}

\subimport{./src}{conclusion}

\postextual{}  % end of the work, now follow references, appendices, etc.

% adding the bibliography to ./src/after.tex leads to compilation issues
% that have to do with the import package
\bibliography{references}

\subimport{./src}{after}

\end{document}
